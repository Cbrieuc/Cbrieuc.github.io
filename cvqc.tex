\documentclass[a4paper,10pt]{article}

%A Few Useful Packages
\usepackage{marvosym}
\usepackage{fontspec} 					%for loading fonts
\usepackage{xunicode,xltxtra,url,parskip} 	%other packages for formatting
\RequirePackage{color,graphicx}
\usepackage[usenames,dvipsnames]{xcolor}
\usepackage[big]{layaureo} 				%better formatting of the A4 page
% an alternative to Layaureo can be ** \usepackage{fullpage} **
\usepackage{supertabular} 				%for Grades
\usepackage{titlesec}					%custom \section

%Setup hyperref package, and colours for links
\usepackage{hyperref}
\definecolor{linkcolour}{rgb}{0,0.2,0.6}
\hypersetup{colorlinks,breaklinks,urlcolor=linkcolour, linkcolor=linkcolour}

%FONTS
\defaultfontfeatures{Mapping=tex-text}
%\setmainfont[SmallCapsFont = Fontin SmallCaps]{Fontin}
\setmainfont[
SmallCapsFont = Fontin-SmallCaps.otf,
BoldFont = Fontin-Bold.otf,
ItalicFont = Fontin-Italic.otf
]
{Fontin.otf}

\titleformat{\section}{\Large\scshape\raggedright}{}{0em}{}[\titlerule]
\titlespacing{\section}{0pt}{3pt}{3pt}
%Tweak a bit the top margin
%\addtolength{\voffset}{-1.3cm}


%--------------------BEGIN DOCUMENT----------------------
\begin{document}

%WATERMARK TEST [**not part of a CV**]---------------
%\font\wm=''Baskerville:color=787878'' at 8pt
%\font\wmweb=''Baskerville:color=FF1493'' at 8pt
%{\wm
%	\begin{textblock}{1}(0,0)
%		\rotatebox{-90}{\parbox{500mm}{
%			Typeset by Alessandro Plasmati with \XeTeX\  \today\ for
%			{\wmweb \href{http://www.aleplasmati.comuv.com}{aleplasmati.comuv.com}}
%		}
%	}
%	\end{textblock}
%}

\pagestyle{empty} % non-numbered pages

\font\fb=''[cmr10]'' %for use with \LaTeX command

%--------------------TITLE-------------
\par{\centering
		{\Huge Brieuc \textsc{Couillerot}
	}\bigskip\par}

%--------------------SECTIONS-----------------------------------
%Section: Personal Data
\section{Informations}

\begin{tabular}{rl}
    \textsc{Téléphone:}     & +1 (438) 405 4626\\
    \textsc{email:}     & \href{mailto:couillerot.brieuc@gmail.com}{couillerot.brieuc@gmail.com}
\end{tabular}


%Section: Work Experience at the top
\section{Expérience professionnelle}
\begin{tabular}{r|p{11cm}}

 \emph & Technicien Pêcheries et Faune Sauvage pour \textsc{Olympic National Park} et \textsc{Lower Elwha Klallam Tribe}, Etats-Unis (WA) \\\textsc{AVRIL - OCTOBRE 2016}&\emph{Superviseurs: Samuel Brenkman, Patrick Crain, Kim Sager-Fradkin }\\&\footnotesize{Contribution à plusieurs projets dans le contexte d'une restauration écosystémique à grande échelle : Etude de la distribution spatiale et temporelle de plusieurs espèces de salmonidés (principalement \textit{Oncorhynchus sp}.). Capture et suivi de populations de cerfs à queue noire et de wapitis de Roosevelt (\textit{Odocoileus hemionus} et \textit{Cervus canadensis roosevelti}), Capture-Marquage-Recapture dans le cadre d'une étude visant à estimer l'impact des communautés de rongeurs sur la dynamique de revégétation d'un ancien lac (principalement \textit{Microtus sp.} et \textit{Peromyscus sp.}) et autres projets satellites (Récupération de données sur des populations de Pékan (Martes pennanti) via pièges photographiques, participation active à différents évènements organisés par l'Olympic National Park).  }\\\multicolumn{2}{c}{} \\

 \textsc{MARS - SEPTEMBRE 2015} & Chargé d'étude chez Centre National de la Recherche Scientifique (CNRS) - Laboratoire "Ecologie, Systématique, Evolution", Orsay, France \\&\emph{Superviseur: François Chiron}\\&\footnotesize{Développement d'un observatoire de la biodiversité du socio-écosystème du Plateau de Saclay dans le contexte d'un territoire partagé entre activités agricoles et urbanisation: Définition des enjeux et objectifs de l'observatoire, identification et communication avec les différents acteurs de la biodiversité de l'échelle locale à l'échelle nationale, acquisition des données de biodiversité et développement fonctionnel de l'observatoire. Rédaction d'un mémoire scientifique servant de base de communication future avec les acters potentiels de l'observatoire.}\\\multicolumn{2}{c}{} \\

\textsc{FEVRIER - AVRIL 2014} & Stagiaire Recherche chez Centre National de la Recherche Scientifique (CNRS) - Laboratoire "Ecologie, Systématique, Evolution", Orsay, France \\&\emph{Superviseur: Carmen Bessa-Gomes}\\&\footnotesize{Etude de l'impact du degré d'urbanisation sur l'utilisation des mangeoires par les communautés aviaires: Installation et maintenance d'un protocole d'échantillonnage basé sur l'utilisation de pièges photographiques; identification des oiseaux capturés par les pièges photographiques, analyse statistique des données et rédaction d'un mémoire scientifique.}\\\multicolumn{2}{c}{} \\

\textsc{JUIN - AOUT 2013} & Technicien Recherche chez Centre National de la Recherche Scientifique (CNRS) - Station Biologique de Foljuif, France \\&\emph{Superviseur: Lydie Blottière}\\&\footnotesize{Etude de l'impact de la température et des perturbations physiques sur les réseaux trophiques aquatiques, travail sur un protocole d'expérimentation en mésocosme aquatique: Récupération d'échantillons d'eau et analyses physico-chimiques et biochimiques, synthèse bibliographique sur l'impact de l'apport de nutriments phosphorés et azotés en milieu aquatique dans le contexte d'un risque d'efflorescence cyanobactérienne}\\\multicolumn{2}{c}{} \\
\end{tabular}

%Section: Education
\section{Education}
\begin{tabular}{1|p{11cm}}
2013 - 2015 & Maîtrise (Deuxième cycle) en \textsc{Environnement, Biologie des Ecosystèmes} à \textbf{Université Paris-Sud-XI}, France\\&\footnotesize{
Ecologie Fonctionelle | Ecologie Evolutive | Modélisation | Dynamiques de Populations | Biostatistiques | Conservation | Socio-Ecosystèmes | Bases de Données | Limnologie Intégrative | Méthodes et Mesures en Ecologie | Droit de l'Environnement}\\\multicolumn{2}{c}{}
\\

2011 - 2013& Licence (Premier cycle) en \textsc{Biologie Générale} à \textbf{Université Paris-Sud-XI}, France\\&\footnotesize{
Ecologie Fonctionnelle | Biologie Animale et Systématique | Biologie Végétale et Systématique | Ecotoxicologie | Pédologie | Géologie | Biologie Moléculaire | Biochimie | Développement Animal | Génétique | Physiologie Animale | Neuro-biologie | Mathématiques}\\\multicolumn{2}{c}{}
\\

2009 - 2011& Classes préparatoires scientifiques à \textbf{Ecole d'Ingénieurs Bio-Industrie (EBI)}, Cergy, France\\&\footnotesize{
Biologie et Chimie des Ecosystèmes | Biochimie Structurelle et Metabolique | Biologie Cellulaire et Génétique | Enzymologie et Cinétique Biochimique | Microbiologie | Biologie Moléculaire | Thermodynamique | Chimie Générale | Électromagnetisme et Ondes | Mécanique des Fluides}\\\multicolumn{2}{c}{}
\\
\end{tabular}


%Section: Langues
\section{Langues}
\begin{tabular}{rl}
 \textsc{Français:}&Langue Natale\\
\textsc{Englais:}&Courant\\
\textsc{Espagnol:}&Basique\\
\textsc{Japonais:}&Débutant\\
\end{tabular}

\section{Compétences en Informatique}
\begin{tabular}{rl}
Logiciels et Plateformes:& \textsc{QGIS}, \textsc{Mark}, \textsc{Microsoft Office}, \textsc{Atom}, \textsc{Github}\\
Languages:& \textsc{R}, \textsc{HTML}, \textsc{CSS}, \textsc{JavaScript}, \textsc{LaTeX}\\
\end{tabular}

\section{Intérêts et Activités}
Technologie, Développement web (site personnel : \href{https://cbrieuc.github.io/index2.html}{https://cbrieuc.github.io/index2.html})\\
Trekking, Kayak, Guitare Classique et autres instruments.\\
Voyages et langues étrangères: Amérique du Nord, Asie, Europe

%\newpage
%\hypertarget{gmat}{\textsc{Gmat}\setmainfont{LMRoman10 Regular}\textregistered\setmainfont[SmallCapsFont=Fontin-SmallCaps]{Fontin-Regular}}

%\XeTeXpdffile ''GMAT.pdf'' page 1 scaled 800

\end{document}
